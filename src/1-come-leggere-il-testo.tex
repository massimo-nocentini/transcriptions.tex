
\chapter{Come leggere il testo}

\footnote{\url{https://media.vaticannews.va/media/audio/program/672/incontri_300820.mp3}}

Iniziamo questo viaggio che è all'interno della prima sezione di testi dell'intera Bibbia, siamo nel libro della Genesi, la prima grande parte di testo di tutta la scrittura. Vogliamo iniziare questa avventura che sarà sulla scoperta della funzione, della luce, dei regali, dei doni, delle grazie nascosti in questo testo partendo da un'introduzione un po' generale. 

Diciamo delle cose di fondo per capirci un po' su come lavorare e come accogliere questa realtà: esistono tante meravigliose letture, tanti tesori da trarre, tante prospettive, però dobbiamo dire qualcosa di fondo. 

Sarebbe molto interessante andare un po' a rivedere quello che dice il Concilio Vaticano II nella bellissima \textit{Dei Verbum} dove si parla di che cos'è la scrittura, come nasce, come va letta, come va accolta; diciamo delle cose adesso per poter prendere in mano l'argomento.

I testi della scrittura vanno letti per quello che sono, non posso avvicinarmi a un testo della scrittura come fosse una cronaca esatta giornalistica, o come fosse un testo filosofico, o come fosse una favola, o altri tipi di approcci al testo. Il testo è quello che è. Ha una peculiarità: innanzitutto è un incontro fra Dio e l'uomo. Cioè Dio parla all'uomo con le parole dell'uomo, cioè Dio usa le parole dell'uomo per rivelarsi, per entrare in comunicazione.

Allora, perché in un certo senso esiste questa cosa che noi chiamiamo rivelazione? Cioè un velo che si apre e qualcuno che si mostra. Dio entra nella nostra vita come si entra nella vita di chiunque, cioè mi rivedo, io entro in comunicazione con qualcuno. In una qualche maniera, per un mezzo che può essere di tanti tipi, io comunque entro in contatto con l'altro e stabilisco appunto uno scambio, una relazione.

\newpage

la rivelazione è per la relazione e però la parola di Dio la Bibbia noi possiamo leggerla per trarne tante cose quale sarà la forma più efficace quella che ci produrrà più frutto che ci permetterà di goderne un po' di più Ecco capire una cosa la Bibbia non racconta le storie in maniera
Ma ripeto giornalistica con cronache di precisione e di esattezza inconfutabile Non è questo il suo interesse la Bibbia nasce cresce si codifica all'interno di una cosa che si chiama liturgia all'interno di un popolo che porta una tradizione e ha un perché ha uno scopo Noi abbiamo delle descrizioni della parola stessa all'interno della parola per esempio c'è il capitolo 55esimo del profeta Isaia che parla di questa parola che Dio emette e che non tornerà lui senza effetto come un seme che feconda la terra il Signore Gesù nel Vangelo di Matteo ha capitolo tredicesimo e il più testo più antico ancora nel Vangelo di Marco a capitolo quarto Porrà la Parabola del seminatore come il proprio il paradigma della rivelazione di Dio del del suo portale
una parola il figlio dell'uomo che porta la parola semina la parola e la parola ha una serie di risultati quindi noi dobbiamo un po' capire perché riceviamo questa parola noi affronteremo i primi 11 capitoli della Genesi e dobbiamo chiederci per poter leggere in maniera anche sintetica perché sarà assolutamente impossibile notare tutto quello che c'è nel testo come li dobbiamo accogliere ribadisco che esistono varie letture e la tradizione Cristiana stabilisce anche dei parametri tradizionali appunto di lettura ma noi li accoglieremo secondo quello che è il suo uso esistenziale ovverosia
la parola di Dio Valletta nella fede non Valletta a prescindere dalla Fede perché è una parola che viene prodotta da un occhi orecchie mani cuori che conoscono il Dio di Israele e nella fede del dio di Israele leggono la realtà e vengono dati a noi perché
 perché noi possiamo entrare nella nostra singola personale e comunitaria vita sulla base di questo paradigma ovverosia Dobbiamo capire che cos'è appunto un paradigma Noi ricordiamo il trauma di ferofers sulla tufer se abbiamo studiato il latino no ho il paradigma totalmente irregolare che resta nell'immaginario abbiamo tutti i paradigmi latini e sappiamo che cosa che il paradigma è liofilizzato L'essenziale di un verbo che deve essere coniugato questo si deve sposare Questo è un tipo che deve trovare moglie e la moglie di Del paradigma della Bibbia è la nostra vita poi così noi abbiamo da coniugare la parola di Dio con la nostra esistenza la nostra vita singolare e comunitaria ecclesiale e personale quindi non possiamo
avvicinarci a questi testi se non per leggere la nostra vita e allora arriviamo al punto che questi testi non vanno letti bisogna Paradossalmente Farsi leggere da questi testi non siamo noi che leggiamo la scrittura e la scrittura che legge noi
allora per farci leggere da questi testi vediamo un pochino di Che testi parliamo questa avventura verterà sui primi 11 capitoli della Genesi perché scegliamo questi capitoli perché hanno una funzione un po' peculiare e anche perché rappresentano una sezione ben precisa noi potremmo prendere altre sezioni di testo Nella Bibbia che hanno delle funzioni e chissà forse un giorno lo faremo Ma in questo caso questa sezione di testo ha una particolarità e prima del capitolo XII dice Beh questo era semplicemente matematico Sì ma il capitolo XII ho capito ben particolare è l'inizio dell'avventura di
un personaggio che viene accompagnato Noi andremo fino a Capitolo 24 dal 12 al 24 sarà tutto su Abramo il quale viene accompagnato fino alla morte e deve vivere un'avventura E questa avventura è peculiare lui è il vero primo chiamato è il padre di tutto quello che verrà e il patriarca per eccellenza è il paradigma della fede e potremmo dire tante altre cose di Abramo Ma è chiaro che dal capitolo 12 il tono del testo cambia radicalmente E che cosa c'è prima che cosa abbiamo noi Nei testi precedenti vediamo un pochino velocemente Di cosa parlano questi capitoli noi avremo nel capitolo primo e nel capitolo secondo due narrazioni non speculari ma parallele della creazione secondo due ottiche ed è meraviglioso che questo sia così perché noi è come se vedessimo la creazione del mondo e poi ce la raccontiamo un'altra volta secondo un'altra prospettiva
positiva quindi comparirà Il mostruoso grandioso inarrivabile capitolo III che sarà il racconto della trasgressione e qualcosa che c'è di così quotidiano per tutti noi in quel testo di così consono alla lettura della nostra avventura personale Quindi arriveremo a vedere le conseguenze tutto questo e vedere questa drammatica storia di Caino e Abele al capitolo quarto e questa storia è la storia del primo omicidio è la storia addirittura di un fratricidio quindi vedremo una degenerazione che accompagna il capitolo quinto l'inizio del capitolo Sesto e che l'aumento della violenza a partire dal primo omicidio tutto quanto diventerà terribile diventerà
questo mondo agghiacciante un mondo che è degenerato per l'appunto e quindi ci sarà la storia se vogliamo in un certo senso enigmatica inquietante e molto profonda del diluvio che prenderai capito di sei sette e otto il diluvio e la sua risoluzione e in un certo senso questo nuovo Adamo che è Noè un nuovo inizio tutto viene come resettato da capo quindi arriviamo alla storia dei figli di Noè e a ciò che rappresenta una nuova ride generazione che implica l'arrivo del capitolo 10 la storia delle tirannie di una situazione che è la preparazione di un ultimo drammatico capitolo di errori umani che sarà il capitolo 11° che
Eh la torre di Babele e la dispersione dell'uomo la sua frammentazione la perdita di ogni comunione di Ogni unità Eccolo alla fine del capitolo 11° comparirà un signor terac che ha un figlio che si chiama Abramo e lì partirà una nuova storia diversa che ormai non conoscerà più questi azzeramenti così drammatici anche se evocativi perché sappiamo che a livello della relazione del testo una cosa come il diluvio rappresenta un'esperienza che poi Israele farà molto più tardi l'esperienza per esempio dell'esilio cioè di questo Azzeramento della sua vita tanti secoli più tardi quando il popolo di Israele perderà il rapporto con Dio si rovinerà e vivrà quel momento terribile che è un po' analogico al discorso drammatico del diluvio Ecco questi sono
sono i testi di cui ci dovremo occupare che hanno la loro bellezza hanno la loro grandiosità hanno la loro drammaticità si potrebbe pensare che c'è una prevalenza di un tema negativo Però se noi andiamo a vedere la verdura così positiva luminosa propositiva grandiosa i primi capitoli con il racconto duplice della creazione è comunque una chiave costruttiva in cui vanno letti un po' tutti i testi
se questi sono i testi con cui dovremmo combattere Che testi sono cioè E perché dedicare delle trasmissioni a questa realtà Beh Dobbiamo un pochino prendere coscienza che questi testi appartengono a una fase diciamo così preistorica in un certo senso quasi pre narrativa non sono vere e proprie narrazioni
nello stesso tempo mantengono una loro verità e una loro concretezza una loro autenticità e anche un'adesione al reale ma non secondo i nostri parametri noi crediamo che le cose sono vere solamente se sono esatte secondo una mentalità post positivista per cui solamente ciò che è misurato col centimetro solamente ciò che è pesato con la bilancia è vero Ma sì è giusto ma anche no perché ci sono tante cose che hanno un altro tipo di verità forse più profonda forse non corredata dalla esattezza che noi amiamo tanto Però forse dotata di maggior profondità di Maggiore autenticità Allora noi dobbiamo capire una cosa Come non vanno letti questi testi non vanno letti come favole pure e semplici
una favola ci dà una morale quando noi sentiamo una favola già dall'inizio pensiamo va bene Vorrà andare a parare da qualche parte Siccome ci abbiamo il paradigma di Fedro e tutte le tutte le varie favole importanti storicamente Noi dovremmo capire un valore qualcosa una morale
 e questo è una tendenza astratta che può andare bene per una favola che peraltro è per l'immaginazione dei bimbi che hanno bisogno di alimentare di analogie di elementi straordinari le cose
Ma non è questo il senso di questi testi questi testi pur se si presenta un serpente che parla e questo Logicamente non è un elemento che pretende di essere un fatto attuale ma non è una favola perché non tende una morale Perché di fatto parlerà sbatterà inciderà sulla concretezza delle cose sulla nostra vita reale avrà a che fare un qualche cosa di tremendamente quotidiano odierno
nello stesso tempo non saranno una vivisezione del reale non saranno una anatomia del reale Cioè non è che ci daranno gli elementi esattamente così come sono ma il loro senso Da dove parte questo tipo di testo c'è da due partono i capitoli uno 11 della Genesi partono da delle domande come abbiamo detto la storia comincia con la storia di Abramo Ma l'uomo andamento si ferma così gli studi biblici hanno rinvenuto come come probabile che la riflessione del Popolo d'Israele si ferma un aumento della sua storia e guarda indietro È come un adulto che posseduta la sua storia al momento se ne chiede il senso
il valore il perché si metterà a cercare A cosa serve questo come come funzionerà la mia vita sulla base di quello che capisco quale è l'origine il senso la base di tutto quello che vivo in fondo mi farò delle domande attenzione a non pensare che queste domande sono delle domande assolutizzanti No sono domande specifiche che però piano piano mi fanno ricostruire il quadro dell'assoluto della mia vita mi spiego è chiaro che noi capiremo tante cose attraverso il capitolo primo e capiremo perché il mondo è fatto come è fatto ma in una chiave esistenziale non filosofica in una chiave utile a vivere oggi non per rispondere a tutte le domande che ho sul creato E così anche la sorgente delle cose
questo capitolo secondo perché ci sarà una differenza proprio testuale fra la maniera di intendere il primo e il secondo racconto della creazione e così mi interrogherò sul male umano sul male che l'uomo fa Ma perché l'uomo fa in male ma perché in un mondo tanto bello Si può fare qualcosa di orribile Perché l'uomo può degenerarsi e diventare assurdo sbagliato perché l'uomo si rovina la vita E perché tante volte la vita diventa una amarezza e il matrimonio può diventare difficile e lavorare duro e crescere dei figli che generarli può essere una fatica immensa perché Israele si chiede il perché delle cose e inizia a rispondere per mezzo di storie che sono lettura della realtà partendo dalla realtà non per descrivere
essere pedissequamente come il mondo è creato E cosa è esattamente successo ma a partire dalla realtà a partire dalle cose e sulla base di una memoria profonda e autentica rispondere a queste domande con dei racconti tecnicamente parliamo di Eziologia Cosa sono gli enzimi eccetera la risposta a domande allora io mi chiederò Ma come mai succede che l'uomo uccida era viene descritto l'omicidio per mezzo di Caino bene ma perché l'uomo vive strutture di potere aggressive e così avremo a che fare con la degenerazione i regni lamec compagnia cantando e sanno spettacoli brutti nello stesso tempo illuminanti e così l'uomo si chiederà ma perché non sappiamo comunicare fra di noi ma perché le persone parlano lingue diverse perché
siamo uniti e così si prende Babele e risponde esistenzialmente alla domanda in maniera narrativa Cioè non risponde concettualmente Questi non sono Greci che cercano le essenze delle cose che si sono ebrei che sono molto fattuali per gli ebrei sono molto più importanti i verbi dei sostantivi i sostantivi che noi ci teniamo tantissimo sono molto meno profondi delle azioni Infatti per un ebreo è molto importante l'agire secondo la legge più che il capire tutta la realtà
siamo ormai verso la fine del tempo di questa trasmissione allora non possiamo che dare una chiave per iniziare a leggere questi testi pensando un pochino proprio al primo testo che ci permetterà di vedere in maniera molto più evidente che negli altri la struttura di come il testo viene fabbricato diciamo così come viene prodotto ovverosia se noi appunto Apriamo la Bibbia cominciamo questa frase in principio Dio creò il cielo la terra la terra riforma deserta le tenebre ricoprivano l'abisso lo spirito di Dio aleggiava sulle acque va avanti racconta della creazione della luce la separazione della luce dalle tenebre il nominare la luce giorno e le tenebre notte E fu sera e fu mattina giorno prima dice la nuova traduzione della Conferenza Episcopale Italiana Ecco questo ritornello tornerà alla fine di ogni
suo interno fino al sesto giorno compreso E fu sera e fu mattina giorno primo giorno secondo via dicendo C'è qui un ritmo inizierà ad incedere di un ritmo di una scansione di un ordine che parlerà di liturgia di ritualità dietro ai nostri testi non ci sta un tizio col computer che si mette a scrivere o con la macchina da scrivere la prima del computer non ci sta un compilatore alla Erasmo e che sa tutto che conosce tutto e capisce tutto no ci sarà una comunità di gente che prega ci sarà il ritmo di qualcuno che ha bisogno di scansioni che Ricordano la ritualità familiare ci sono dei riti Nelle famiglie ci sono dei riti soprattutto bisogna pensare delle
famiglie nomadiche e bisogna pensare un tempo in cui Mica c'era la televisione la sera la gente cosa faceva la sera prima di andare a dormire accendeva nonno Cioè nonno si metteva a raccontare il nonno raccontava E se qualcuno ha mai provato a raccontare una favola a un bambino sa che si hilair con due volte la seconda volta ti trova tutti tutte le differenze che fai Se sbagli una parola No zio Non era così era era cosà e non si può sbagliare la tradizione orale ha una ritualità mettere a letto un bambino e raccontargli una storia ecco parla di quelle ripetitività rassicuranti che sono tipiche della vita familiare Ecco il ritmo il racconto sa di liturgia sa di tradizione sa di ripetizione sa di qualcosa raccontato mille volte mille volte ritrovato vero Così noi dovremo leggere questi testi sulla base di una Sapienza arcaica che
velocità vero ciò che proclama vero sempre lo si può ripetere mille volte reggerà una ripetizione portata avanti per centinaia d'anni e ancora oggi questi testi Li leggiamo e li troviamo veri
 grazie ad un Fabio Rosini Grazie naturalmente a tutti voi per averci seguito non mi resta che augurarvi buon proseguimento all'ascolto di Radio Vaticana Italia
la trama del reale catechesi di don Fabio Rosini sui primi 11 capitoli della Bibbia
 primo incontro Come leggere il testo